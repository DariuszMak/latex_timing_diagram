% Timing Diagram using the 'tikz-timing' package (v0.5 - 2009/05/15).
%
% (c) Copyright 2009 - Martin Scharrer <martin@scharrer-online.de>
%
% This code is open source under the GPLv3 and/or the LPPL v1.3 (or later).
% Please feel free to use, change and copy it for your own benefit.
%

\documentclass{article}
\usepackage{tikz-timing}[2009/05/15]

%%%<
\usepackage{verbatim}
\usepackage[active,tightpage]{preview}
\PreviewEnvironment{tikzpicture}
\setlength\PreviewBorder{5pt}%
%%%>

\begin{document}

\def\degr{${}^\circ$}
\begin{tikztimingtable}
$I_{0}$ & HHHH N(A2) HHHH N(A3) HHHH N(A4) HHHH N(A5) HHHH N(A6) HHHH N(A7) HHHH N(A8) HHHH N(A9) HHHH\\
$I_{1}$ & 18{2H}\\
$I_{2}$ & 18{2L}\\
$I_{3}$ & 18{2H}\\
$OK_{(button)}$ & LLLLCHHHHHHHHHHHHHHHHHHHHHHHHHHHHHHH \\
$C_{(receiver)}$ & LLLL 14{2C} LLLL\\
CBO & LLLL LLLL CHHH CLLL CHHH HHHH HHHH HHHH HHHH\\
CBO delay & LL LLLL LLLL CHHH CLLL CHHH HHHH HHHH HHHH HH\\
$O_{0}$ & 16{2L}CHHH\\
$O_{1}$ & 16{2L}CHHH\\
$O_{2}$ & 18{2L}\\
$O_{3}$ & 16{2L}CHHH\\
Parity bit & LLLL N(B2) LLLL N(B3) LLLL N(B4) LLLL N(B5) LLLL N(B6) LLLL N(B7) LLLL N(B8) LLLL N(B9) CHHH\\
\extracode
  \tablerules
  \begin{pgfonlayer}{background}
    \foreach \n in {2,...,9}
      \draw [help lines] (A\n) -- (B\n);
  \end{pgfonlayer}
\end{tikztimingtable}
%
\end{document}